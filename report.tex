\documentclass{article}
\usepackage{graphicx} % Required for inserting images
\usepackage[margin=1in]{geometry}

\title{Coursework 2: Covid Cases Database}
\author{Hossameldin Tammam}
\date{April 2024}

\begin{document}

\maketitle

\section{Relation Model}
\subsection{EX1}

\begin{tabular}{|c|c|}
 \hline
 Attribute & Data Type \\
 \hline
 dateRep & TEXT\\
 \hline
 day & INTEGER\\
 \hline
 month & INTEGER\\
 \hline
 year & INTEGER\\
 \hline
 cases & INTEGER\\
 \hline
 deaths & INTEGER\\
 \hline
 countriesAndTerritories & TEXT\\
 \hline
 geoId & TEXT\\
 \hline
 countryterritoryCode & TEXT\\
 \hline
 popData2020 & INTEGER\\
 \hline
 continentExp & TEXT\\
 \hline
\end{tabular}

\subsection{EX2}    
\begin{tabular}[t]{|c|}
  \hline
  \textbf{Functional Dependency} \\
  \hline
  dateRep $\to$ day\\ 
  dateRep $\to$ month\\ 
  dateRep $\to$ year\\
  (day,month,year) $\to$ dateRep \\
  dateRep,geoId $\to$ cases\\ 
  dateRep, geoId $\to$ deaths \\ 
  geoId $\to$ countriesAndTerritories\\ 
  geoId $\to$ countryterritoryCode\\ 
  countryterritoryCode $\to$ geoId\\ 
  countryterritoryCode $\to$ popData2020\\ 
  geoId $\to$ popData2020\\ 
  \hline
\end{tabular}

\pagebreak
\subsection{EX3}
\begin{table}[h]
 \begin{tabular}{|c|} 
  \hline
  \textbf{Candidate keys} \\ [1ex]
  \hline 
  dateRep, countriesAndTerritories\\ [1ex]
  dateRep, geoId\\ [1ex]
  dateRep, countryterritoryCode\\ [1ex]
  \hline
 \end{tabular}
\end{table}

\subsection{EX4}
Using dateRep is more convenient than using one of day, month or year, and geoID is perfect as it is unique to each country and short.
\begin{table}[h]
\begin{tabular}{|c|}
\hline
\textbf{Primary key} \\ [1ex]
\hline
dateRep, geoID\\ [1ex] 
\hline
\end{tabular}
\end{table}

\section{Normalisation}
\subsection{EX5}
There are two partial dependencies: \\
\\
dateRep $\to$ day, month, year\\
geoID $\to$ countriesAndTerritories, countryterritoryCode, popData2020, continentExp \\
\\
We will separate these 2 relations and have a table:\\
\\
dateRep, geoID $\to$ cases, deaths

\subsection{EX6}
\begin{verbatim}
CovidDates(\underline{dateRep}, day, month, year)\\
CountryDetails(\underline{geoID}, countriesAndTerritories, 
countryterritoryCode, continentExp)\\
CovidCases(\underline{dateRep, geoID}, cases, deaths, popData2020)\\
\\
The primary key for each table is:\\
CovidDates : dateRep\\
CountryDetails : geoID\\
CovidCases : dateRep, geoID\\
\end{verbatim}

\subsection{EX7}
In the new relations, there are no transitive dependencies

\subsection{EX8}
Both relations are already in 3NF because each non-prime attribute is directly dependent on the keys, and there are no transitive dependencies.

\subsection{EX9}
The relations are already in Boyce-Codd Normal Form (BCNF). BCNF ensures that all functional dependencies are based on candidate keys.

\section{Modelling}
\subsection{EX10}
1. Launch SQLite3 in the directory with your dataset.csv file by typing in "sqlite3 coronavirus.db"\\
2. type in ".mode csv" to enter CSV mode\\
3. type in ".import dataset.csv dataset"\\
4. type in ".output dataset.sql"\\
5. type in ".dump dataset"\\

\subsection{EX11}
Creates all 3 tables stated in EX6 with all the types for all the colomns 
\begin{verbatim}
CREATE TABLE CovidDates (
    dateRep TEXT PRIMARY KEY,
    day INTEGER,
    month INTEGER,
    year INTEGER
);

CREATE TABLE CountryDetails (
    geoId TEXT PRIMARY KEY,
    countriesAndTerritories TEXT,
    countryterritoryCode TEXT,
    continentExp TEXT
);

CREATE TABLE CovidCases (
    dateRep TEXT not null,
    geoId TEXT not null ,
    cases INTEGER,
    deaths INTEGER,
    popData2020 INTEGER,
    primary key (dateRep,geoId)
);
\end{verbatim}

\subsection{EX12}
Insert all values from the dataset table to the 3 tables created in EX11
\begin{verbatim}
INSERT INTO CovidDates (dateRep, day, month, year)
SELECT DISTINCT dateRep, day, month, year FROM dataset;

INSERT INTO CountryDetails (geoId, countriesAndTerritories, 
countryterritoryCode, continentExp)
SELECT DISTINCT geoId, countriesAndTerritories, countriesAndTerritories, 
continentExp FROM dataset;

INSERT INTO CovidCases (dateRep, geoId, cases, deaths, popData2020)
SELECT DISTINCT dateRep, geoId, cases, deaths, popData2020 FROM dataset;
\end{verbatim}

\subsection{EX13}
Make sure both ex11 and ex12 work on a blank database and it does.
\begin{verbatim}
sqlite3 coronavirus.db < dataset.sql
sqlite3 coronavirus.db < ex11.sql
sqlite3 coronavirus.db < ex12.sql 
\end{verbatim}
\section{Querying}
\subsection{EX14}
Access sum of cases and deaths from CovidCases table
\begin{verbatim}
SELECT sum(cases) as total_cases, sum(deaths) as total_deaths\\
from CovidCases;
\end{verbatim}
\subsection{EX15}
Using DATE() to convert string dateRep to a date so it can be sorted properly, keep only rows with geoID as UK and show the date and cases.
\begin{verbatim}
SELECT dateRep, cases FROM CovidCases WHERE geoId = 'UK'
ORDER BY DATE(dateRep) ASC;
\end{verbatim}
\subsection{EX16}
Joined CovidCases and CountryDetails table to access country names, ordered TotalCases and TotalDeaths by date, used Date() to convert dateRep to a date.
\begin{verbatim}
SELECT CountryDetails.countriesAndTerritories AS Country,
       CovidCases.dateRep AS Date,
       CovidCases.cases AS TotalCases,
       CovidCases.deaths AS TotalDeaths
FROM CountryDetails
JOIN CovidCases ON CountryDetails.geoId = CovidCases.geoId
ORDER BY Date(Date) ASC;

\end{verbatim}
\subsection{EX17}
Joined CountryDetails table to access country names, calculated case and deaths percentage against population and rounded to 2 decimal places.
\begin{verbatim}
SELECT CountryDetails.countriesAndTerritories,
       ROUND((SUM(cases) / popData2020) * 100, 2) AS PercentCases,
       ROUND((SUM(deaths) / popData2020) * 100, 2) AS PercentDeaths
FROM CovidCases
NATURAL JOIN CountryDetails
GROUP BY countriesAndTerritories, popData2020;
\end{verbatim}
\subsection{EX18}
Calculate  percentage of deaths compared to cases and rounded to 2, joined CovidCases and CountryDetails table to access country names, limit to 10 to provide only first 10, order by percentage descending.
\begin{verbatim}
SELECT CountryDetails.countriesAndTerritories AS Country,
    ROUND((SUM(deaths) / sum(cases)) * 100, 2) AS Precentage
FROM CountryDetails
JOIN CovidCases ON CountryDetails.geoId = CovidCases.geoId
GROUP BY CountryDetails.countriesAndTerritories
ORDER BY Precentage DESC
LIMIT 10;
\end{verbatim}
\subsection{EX19}
For the united kingdom provide all the commulative deaths and cases and sort it by date.
\begin{verbatim}
SELECT dateRep as Date,
	SUM(deaths) OVER (ROWS UNBOUNDED PRECEDING) AS CumulativeUKDeaths,
	SUM(cases) OVER (ROWS UNBOUNDED PRECEDING) AS CumulativeUKCases
FROM CovidCases
WHERE geoId = 'UK'
ORDER BY date(dateRep);
\end{verbatim}
\section{Extension}
\subsection{EX20}
Have Not Attempted Extension
\end{document}
